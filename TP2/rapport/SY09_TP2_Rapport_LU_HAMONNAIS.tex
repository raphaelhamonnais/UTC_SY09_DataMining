\documentclass[a4paper,10pt]{report}

\usepackage{packages/rapportutc}
%\usepackage{packages/include-packages}


%%%%%%%% Références %%%%%%%%
\usepackage{cleveref}
\usepackage{hyperref}
\usepackage[nottoc, notlof, notlot]{tocbibind} % bibliographie http://tex.stackexchange.com/questions/71129/bibliography-in-table-of-contents
\usepackage{natbib} % bibliographie

%%%%%%%% Code informatique %%%%%%%%
\usepackage{packages/Sweave} %package d'affichage des codes R
\usepackage{listings} % pour hightlight code

%%%%%%%% Formules mathématiques %%%%%%%%
\usepackage{amsmath, amsthm, amssymb, graphics, setspace} %packages de mathématiques
\usepackage{chemist} %formule chimique 

%%%%%%%% Mise en forme %%%%%%%%
% Mise en forme graphique
\usepackage{graphicx,wrapfig,lipsum} % pour afficher des figures à côté du texte
\usepackage[linewidth=1pt]{mdframed} % permet de générer et gérer des frames
\usepackage{rotating} % rotations on tables, captions, text, ...
% Mise en forme images et tableaux
\usepackage{float} % permet de spécifier l'option "H" aux captions afin de les positionner de manière fixe
\usepackage{subcaption} % permet d'afficher plusieurs images dans une caption
\usepackage{array} % meilleurs "table" et "tabular"
% Mise en forme texte
\usepackage{setspace} % permet de spécifier l'espacement interligne
\usepackage{ulem} % \sout{Texte à barrer} \xout{Texte à hachurer} \uwave{Texte à souligner par une vaguelette}
\usepackage{calc,enumitem}  % Mise en forme l'environnement itemsize description etc.
\usepackage{color} % utilisation de couleurs
\usepackage{ae,aecompl} % Vir­tual fonts for T1 en­coded CMR-fonts
\usepackage{pifont} % com­mands for Pi fonts (Ding­bats, Sym­bol, etc.)
\usepackage{comment} % Selectively include/exclude portions of text \comment....\endcomment

\onehalfspacing % espacement interligne
\setlength{\parindent}{.5em} % indentation des retraits de première ligne

%%%%%%%%%%%%%%%%%%%%%%%%%%%%%%%%%%%%%%%%%%%%%%%%%%%%%%%%%%%%%%%%%%%%%%%%%%%% 

\title{TP 2 - }
\author{LU Han - HAMONNAIS Raphaël}
\date{\today}

\uv{SY09}
\branche{Génie Informatique}
\filiere{Fouille de Données et Décisionnel}
%%%%%%%%%%%%%%%%%%%%%%%%%%%%%%%%%%%%%%%%%%%%%%%%%%%%%%%%%%%%%%%%%%%%%%%%%%%%

\begin{document}

\renewcommand{\labelitemi}{\large\textcolor{tatoebagreen}{\fg}}
\newgeometry{top=2.5cm,bottom=2cm,left=2cm,right=2cm}
\groovypdtitre
\restoregeometry % restaure la géométrie par défaut de latex

%%%%%%%%%%%%%%%%%%%%%%%%%%%%%%%%%%%%%%%%%%%%%%%%%%%%%%%%%%%%%%%%%%%%%%%%%%%% 

\tableofcontents

%%%%%%%%%%%%%%%%%%%%%%%%%%%%%%%%%%%%%%%%%%%%%%%%%%%%%%%%%%%%%%%%%%%%%%%%%%%%



\chapter{Visualisation des données}

\section{Données Iris}

\begin{figure}[H]

\end{figure}


\begin{figure}[H]
	\centering
	\captionsetup{justification=centering, margin=2cm}
	\begin{subfigure}[b]{0.5\linewidth}
		\centering
		\captionsetup{justification=centering, margin=2cm}
		\includegraphics[width=0.3\linewidth]{img/}
		\caption{\scriptsize Données \texttt{Iris} dans le premier plan factoriel sans tenir compte de l'espèce.}
		\label{fig:}
	\end{subfigure}%
	\begin{subfigure}[b]{0.5\linewidth}
		\centering
		\captionsetup{justification=centering, margin=2cm}
		\includegraphics[width=0.3\linewidth]{img/}
		\caption{\scriptsize Données \texttt{Iris} dans le premier plan factoriel en tenant compte de l'espèce.}
		\label{fig:}
	\end{subfigure}%
	\caption{
		\scriptsize Représentation des données \texttt{Iris} dans le premier plan factoriel.
	}
	\label{fig:tab_effectifs_et_contingence_resultats_diplome_origine}%
\end{figure}

Affichage dans le premier plan factoriel après ACP sans tenir compte de l'espèce~:

On observe deux groupes bien distincts.

Affichage premier plan factoriel en tenant compte de l'espèce

On voit qu'un des deux groupes est en fait constitué de deux des espèces différentes. 
On obtient donc deux informations précieuses~: les méthodes de classification géométriques tendront à nous donner deux classes au lieu de trois. Il faudra donc spécifier qu'on recherche bien 3 classes et non 2.


\section{Données Crabs}

Affichage premier plan factoriel sans tenir compte de l'espèce ou du sexe

On observe deux groupes bien distincts.


Affichage premier plan factoriel en tenant compte de l'espèce et du sexe

On constate que les deux groupes observés précédemment correspondent à l'espèce des crabes. On voit aussi apparaître deux autres groupes au sein des premiers qui délimitent le sexe.
On va donc chercher à faire une classification à 4 classes. On note quand même que la délimitation entre les sexes est plus floue que celle entre les espèces.


\section{Données Mutation}

\end{document}